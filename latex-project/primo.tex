
% Esempio di scrittura con LaTeX.
\documentclass[a4paper,11pt,twoside]{article}
%\usepackage{geometry}
%\geometry{a4paper,top=0.5cm,bottom=1cm,left=2cm,right=0.5cm}
\usepackage[T1]{fontenc}         % Codifica dei font.
\usepackage[utf8]{inputenc}      % Lettere accentate.
\usepackage[italian]{babel}      % Lingua del documento.
\usepackage{lipsum}              % Generatore di testo casuale.
\usepackage{url}
\pagestyle{plain}                % Numeri a piè di pagina.

\begin{document}
	\author{Mauro Quaglia}
	\title{Q-\LaTeX}
	\maketitle %Serve per rendere visibile i due precedenti.
	
	\begin{abstract}
		\lipsum[1]
	\end{abstract}
	
	\tableofcontents
	
	\section{Premessa}
	Qui si inizia
	
	\section{Capitolo 1}
	Qui si continua\dots
	
	\section{Capitolo 2}
	\label{sec:esempio}
	
	\subsection{Un sottoparagrafo}
	
	\section{Fine}
	Qui si finisce - sono a pagina 53
	
	% Bibliografia.
	\begin{thebibliography}{9}
		\bibitem{quaglia:arte} The Quail arth
		\emph{L'arte dello yoga}
		\url{http://www.trovaprezzi.it/}
	\end{thebibliography}
\end{document}

